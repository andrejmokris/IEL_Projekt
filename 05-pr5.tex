\section{Příklad 5}
% Jako parametr zadejte skupinu (A-H)
\patyZadani{D}

\begin{center}
    \textbf{Krok 1} - Opíšeme obvod rovnicou pomocou Ohmovho zákona a II. Kirchhoffového zákona
\end{center}

\begin{gather*}
    U_R + U_R - U = 0 V \\
    R \times i + U_C - U = 0 V \\
    I = \frac {U_{R}} {R} = \frac {U - U_{C}} {R} \\
\end{gather*}

%u'_c

\begin{center}
    \textbf{Krok 2} - Zostavíme rovnicu ${u'_{c}}$
\end{center}

\begin{gather*}
    u'_{c} = \frac {i} {c} = \frac {\frac {U - U_{C}} {R}} {c} = \frac {U - U_{C}} {R \times C} = \frac  {35 - U_{C}} {5 \times 25} \\
\end{gather*}
 

\begin{center}
    \textbf{Krok 3} - Zostavíme diferenciálnu rovnicu
\end{center}

\begin{gather*}
    u'_{c} + U_{C} \times \frac {1} {125} = \frac {7} {25} \\
\end{gather*}

\newpage 

\begin{center}
    \textbf{Krok 4} - Riešenie charakteristickej rovnice pre ${\lambda}$
\end{center}

\begin{gather*}
    \lambda + \frac {1} {R \times C} = 0 \\\
    \lambda + \frac {1} {125} = 0 \\\
    \lambda = - \frac {1} {125} \
\end{gather*}

\begin{center}
    \textbf{Krok 5} - Očakávané riešenie
\end{center}

\begin{gather*}
    u_{c}(t) = k(t) \times e^{\lambda \times t} \\
    u_{c}(t) = k(t) \times e^{-\frac {t}{125}}
\end{gather*}

\begin{center}
    \textbf{Krok 6} - Dosadíme do všeobecnej rovnice a zderivujeme
\end{center}

\begin{gather*}
    u'_{c}(t) = k'(t) \times e^{-\frac {t}{125}} + k(t)(-\frac {1} {125})e^{-\frac {t}{125}}
\end{gather*}

\begin{center}
    \textbf{Krok 7} - Dosadíme do diferenciálnej rovnice z kroku 3
\end{center}

\begin{gather*}
    u'_{c} + U_{C} \times \frac {1} {125} = \frac {7} {25} \\
   k'(t) \times e^{-\frac {t}{125}} + k(t)(-\frac {1} {125})e^{-\frac {t}{125}} + U_{C} \times \frac {1} {125} = \frac {7} {25} \\
    k'(t) \times e^{\frac {t}{125}} = 0 \\
\end{gather*}

\begin{center}
    \textbf{Krok 8} - Zbavíme sa derivacie pomocou intergrácie
\end{center}

\begin{gather*}
    k'(t) = \frac {7} {25} \times e^{\frac {t}{125}} \\
    \int k'(t) = \int \frac {7} {25} e^{\frac {t}{125}} \\
    k(t) = 35e^{\frac{t}{125}} + K
\end{gather*}

\newpage

\begin{center}
    \textbf{Krok 9} - Dosadime do očakávaného riešenia
\end{center}

\begin{gather*}
   u_{c}(t) = k(t) \times e^{-\frac {t}{125}} \\ 
   u_{c}(t) = (35e^{\frac{t}{125}} + K) \times e^{-\frac {t}{125}} \\ 
    u_{c}(t) = 35 + Ke^{-\frac{1}{125}}
\end{gather*}

\begin{center}
    \textbf{Krok 10} - Dosadíme počiatočnú podmienku
\end{center}

\begin{gather*}
   u_{c} = 35e^{\frac{t}{125}} + K \\
   15 = 35 + K \\
    K = - 20
\end{gather*}


Výsledok:

\begin{gather*}
	u_{c}(t) = 35-20e^{-\frac{t}{125}}
\end{gather*}

\begin{center}
    \textbf{Krok 11} - Skúška správnosti riešenia 
\end{center}

\begin{gather*}
    u'_{c} + U_{C} \times \frac {1} {125} = \frac {7} {25} \\
    u'_{c} + \frac{1}{125} \times (35-20e^{-\frac{1}{125}}) = \frac {7} {25}\\
    u'_{c} + \frac {35}{125} - \frac {20e^{- \frac {t}{125}}} {125} = \frac {7} {25} \\\
    u'_{c} = \frac {20e^{- \frac {t}{125}}} {125} \\
    u'_{c} + U_{C} \times \frac {1} {125} = \frac {7} {25} \\
    \frac {20e^{- \frac {t}{125}}} {125} + (35-20e^{-\frac{t}{125}}) \times \frac {1}{125} = \frac {7}{25} \\
    \frac {4e^{- \frac {t}{125}}} {25} + (7-4e^{-\frac{t}{125}}) \times \frac {1}{25} = \frac {7}{25} \\
    \frac {4e^{- \frac {t}{125}}} {25} + \frac {7}{25} - \frac {4e^{-\frac{t}{125}}} {25} = \frac {7}{25} \\
    \textbf{0 = 0} \\
\end{gather*}


